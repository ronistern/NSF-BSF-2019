\documentclass[12pt]{article}

\usepackage[ruled,vlined,linesnumbered]{algorithm2e}

\usepackage{times}
\usepackage{anysize}
\marginsize{2cm}{2cm}{2cm}{3cm}
%\onehalfspace
%\doublespace
\setlength{\parindent}{0.8cm}
%\setlength{\parskip}{0.2\baselineskip}
%\setlength{\topmargin}{2cm}
%\setlength{\textheight}{25cm}
%\setlength{\textwidth}{14cm}
%\setlength{\oddsidemargin}{2cm}
%\setlength{\evensidemargin}{2cm}
\usepackage{fancyhdr}
\pagestyle{fancy}
\usepackage{url}


\linespread{1}

\lhead{Data-and-Model Driven Reasoning} \rhead{R. Stern and B. Juba}
\cfoot{\thepage} 
%\cfoot{} 
\pagenumbering{arabic}
%\pagenumbering{Roman}

\begin{document}

\section*{Biographical Sketch}

\subsection*{Senior Personnel: Brendan Juba (PI)}
\paragraph{Professional Preparation}
\ \\
\begin{tabular}{lll}
{\em Undergraduate institution:}& & \\
Carnegie Mellon University&Computer Science&B.S., 2005\\
 & & \\
{\em Graduate institutions:}& & \\
Carnegie Mellon University&Mathematical Sciences&M.S., 2005\\
Massachusetts Institute of Technology&Computer Science&Ph.D., 2010\\
 & & \\
{\em Postdoctoral Institutions:}& & \\
MIT \& Harvard University&Computer Science&2010--2012\\
Harvard University&Computer Science&2012--2014
\end{tabular}
\paragraph{Appointments}
\ \\
\begin{tabular}{lrl}
Washington University in St. Louis&Computer Science \& Engineering&\\
&Assistant Professor&2014--present
\end{tabular}
\paragraph{Publications - five most related}
\begin{itemize}
  \item  R.~Stern and B.~Juba.
    \newblock Efficient, Safe, and Probably Approximately Complete Learning of 
              Action Models.
    \newblock In {\em Proceedings of the 26th International Joint Conference on
              Artificial Intelligence, IJCAI 2017}, pages 4405--4411, Melbourne,
              Australia, 2017. 
    \newblock Available at     \url{http://www.cse.wustl.edu/~bjuba/papers/pac-planning-ijcai.pdf}

  \item M.~Zhang, T.~Mathew, and B.~Juba.
    \newblock An Improved Algorithm for Learning to Perform Exception-Tolerant
              Abduction.
    \newblock In {\em Proceedings of the 31st AAAI Conference on Artificial
              Intelligence,} pages 1257--1265, San Francisco, CA, USA, 2017.
    \newblock Available at
      \url{http://www.cse.wustl.edu/~bjuba/papers/zmj-aaai.pdf}
  
  \item B.~Juba.
    \newblock Integrated common sense learning and planning in {POMDPs}.
    \newblock Journal of Machine Learning Research, 17(96):1--37, 2016.
    \newblock Avaliable at 
      \url{http://jmlr.org/papers/v17/13-584.html}.

 \item B.~Juba.
    \newblock Learning abductive reasoning using random examples.
    \newblock In {\em Proceedings of the 30th AAAI Conference on Artificial
              Intelligence}, pages 999--1007, Phoenix, AZ, USA, 2016.
    \newblock Available at \url{http://www.cse.wustl.edu/~bjuba/papers/abduction-aaai.pdf}.

  \item B.~Juba.
    \newblock Implicit learning of common sense for reasoning.
    \newblock {\em (Selected for oral presentation.)}
    \newblock In {\em Proceedings of the 23rd International Joint Conference on 
              Artificial Intelligence, IJCAI 2013}, pages 939--946, Beijing, China, 2013.
    \newblock Available at \url{http://ijcai.org/Proceedings/13/Papers/144.pdf}
\end{itemize}
\paragraph{Publications - five other}
\begin{itemize}
 \item M.~Chakraborty, K.~P.~Chua, S.~Das, and B.~Juba. 
    \newblock Coordinated Versus Decentralized Exploration In Multi-Agent 
              Multi-Armed Bandits.
    \newblock In {\em Proceedings of the 26th International Joint Converence on 
              Artificial Intelligence, IJCAI 2017}, pages 164--170, Melbourne, 
              Australia, 2017. 

 \item B.~Juba.
    \newblock Conditional sparse linear regression.
    \newblock In {\em 8th Innovations in Theoretical Computer Science,}
              LIPIcs volume 67, 2017.
    \newblock Preliminary version: arXiv:1608:05152 [cs.LG]

  \item O.~Goldreich, B.~Juba, and M.~Sudan.
    \newblock A Theory of Goal-Oriented Communication.
    \newblock Journal of the ACM. 59(2), Article 8, 2012.
    \newblock Electronic version: \url{http://eccc.uni-trier.de/report/2009/075/}
    
  \item B. Juba, {\em Universal Semantic Communication}, Springer, Berlin, 2011.
  
    \item B.~Juba and M.~Sudan.
    \newblock Universal Semantic Communication I.
    \newblock In {\em Proceedings of the 2008 ACM International Symposium on 
              Theory of Computing (STOC)}, pages 123--132, Victoria, BC, Canada, 2008.

\end{itemize}
\paragraph{Synergistic Activities}
\begin{enumerate}
\item The PI co-supervised an undergraduate during summer 2014 (with L. G. 
Valiant and F. Doshi) who worked on using medical ontologies to help extract 
rules that predict symptoms in medical data. He co-supervised another
(female) undergraduate during summer 2015 (with S. Das) who worked on strategies
for multi-armed bandits with multiple, communicating players. During
summer 2016, with R. Pless he supervised two undergraduates (one male, one female) who worked
on attaching semantics from meta-data to ``anomalous'' images from webcams, and one who worked on sum-of-squares lower bounds for
machine learning problems. Finally, during summer 2017, he supervised three undergraduates (one female, one underrepresented minority) who worked on developing new algorithms for the conditional linear regression problem. He has also supervised four other undergraduates who
worked on independent study research projects in learning theory and proof
complexity between Spring 2015 and Fall 2018.
\item The PI developed a graduate course on the theory of AI and machine learning that integrates material on learning theory and automated reasoning with applications to classical AI problems.
\item The PI participated in a workshop for faculty on teaching STEM subjects 
hosted at Washington University June 9--11, 2015, and participated in an 
``alumni day'' of the next workshop in the series on June 16, 2016.
\item The PI has served on conference program committees, including AAMAS 2018, AAAI 2018, ALT 2017, AAMAS 2017 (senior PC), AAAI 2017, IJCAI 2016, ITCS 2016, AAAI 2016, and as a subreferee several times each for STOC, FOCS, and ITCS (and
occasionally for COLT, SODA, CCC, STACS, and others).
The PI has reviewed for J. ACM, Theory
of Computing, Quantum Information \& Computation, Entropy, and IEEE Transactions on
Molecular, Biological, and Multi-Scale Communications.
\item The PI has served on four NSF review panels.
\end{enumerate}

\end{document}